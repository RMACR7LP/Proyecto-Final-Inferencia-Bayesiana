\documentclass[12 pt]{article}
\usepackage[margin=1 in]{geometry}
\usepackage[utf8]{inputenc} 
\usepackage{amsmath, amsfonts, amssymb}
\usepackage{amsthm}
\usepackage{dsfont}
\usepackage[spanish]{babel}
\usepackage[pdftex]{hyperref}
\usepackage{hyperref}
\hypersetup{
    colorlinks=true,
    linkcolor=black,
    citecolor = black,
    filecolor=magenta,      
    urlcolor= blue,
}
\usepackage{enumitem}
\usepackage{graphicx}
\usepackage{dirtytalk}
\usepackage{listings}
\usepackage[usenames,dvipsnames]{color}   \usepackage{dirtytalk} 
\usepackage{float}
\newcommand{\R}{\mathbb{R}}
\newcommand{\N}{\mathbb{N}}
\restylefloat{table}
\newtheorem{thm}{Teorema}
\newtheorem{lema}{Lema}


\lstset{ 
  language=Python,                     
  basicstyle=\scriptsize\ttfamily, 
  numberstyle=\tiny\color{Blue},  
  numbers=left,
  stepnumber=1,                  
  backgroundcolor=\color{white},  
  showspaces=false,               
  showstringspaces=false,         
  showtabs=false,                 
  frame=single,                   
  rulecolor=\color{black},        
  tabsize=3,             
  captionpos=b,                   
  breaklines=true,                
  breakatwhitespace=false,        
  keywordstyle=\color{RoyalBlue},      
  commentstyle=\color{YellowGreen},   
  stringstyle=\color{ForestGreen}     
} 




\title{Proyecto Final}
\author{Alam Rojas\\
Cristian José Alvarez Bran \\
Fernando\\
Ricardo 
}
\date{}



\begin{document}
\maketitle



\section{Inciso i}
Procedemos a calcular $p(m,\theta)$: 
\begin{align*}
    p(m, \theta) &= \int_{0}^{\infty}p(m, \theta , \lambda)d\lambda\\
    &= \int_{0}^{\infty}p(m|\theta, \lambda)p(\theta, \lambda) d\lambda\\
    &= \int_{0}^{\infty}e^{-\frac{\lambda}{\theta}}\frac{\lambda^{m-1}}{m!\theta^m} d\lambda\\
    &=\frac{\theta^m\Gamma(m)}{m!\theta^m}\\
    &= \frac{1}{m-1}.
\end{align*}

\section{Inciso ii}

\begin{lema}
La serie 
\begin{equation*}
    \sum_{m=1}^{\infty}\frac{\theta^S(1-\theta)^{nm-S}}{m}\prod_{i=1}^{n}{{m}\choose{x_i}}
\end{equation*}
converge.
\end{lema}
\begin{proof}
Examinaremos el cociente entre términos consecutivos de la serie:
\begin{align*}
    \frac{a_{m+1}}{a_m} &=  
    \frac{\frac{(1-\theta)^{n(m+1)}}{m+1}\prod_{i=1}^{n}{{m+1}\choose{x_i}}}{ \frac{(1-\theta)^{nm}}{m}\prod_{i=1}^{n}{{m}\choose{x_i}}}\\
    &= \frac{(1-\theta)^{n}m\prod_{i=1}^{n}{{m+1}\choose{x_i}}}{(m+1)\prod_{i=1}^{n}{{m}\choose{x_i}}}.
\end{align*}
Luego notemos que 
\begin{equation*}
    {{m+1}\choose{x_i}} = \frac{(m+1)!}{x_i!(m-x_i+1)!} =  \frac{m+1}{m-x_i+1}\cdot\frac{m!}{x_i!(m-x_i)!} = \frac{m+1}{m-x_i+1}{{m}\choose{x_i}}.
\end{equation*}
Sustituyendo en la expresión anterior llegamos a que 
\begin{equation}
    \frac{a_{m+1}}{a_m} = (1-\theta)^n\frac{m}{m+1}\prod_{i=1}^{n} \frac{m+1}{m-x_i+1}.
\end{equation}
De modo que $\lim_{m\to \infty} \frac{a_{m+1}}{a_m} = (1-\theta)^n < 1$ y por el criterio de la razón, la serie converge.
\end{proof}
\end{document}